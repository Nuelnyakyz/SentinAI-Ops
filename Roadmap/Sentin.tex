\documentclass[11pt]{article}

% Language and encoding
\usepackage[english]{babel}
\usepackage[utf8]{inputenc}
\usepackage[T1]{fontenc}
\usepackage{fontspec}
\setmainfont{Quicksand}
% \renewcommand{\familydefault}{\sfdefault} % Use sans-serif as default
% \usepackage{helvet} % Helvetica font

% Page layout
\usepackage[a4paper, total={6.5in, 9in}]{geometry}

% Math packages
\usepackage{amsmath}
\usepackage{amsfonts}
\usepackage{amssymb}
\usepackage[version=4]{mhchem}
\usepackage{stmaryrd}
\usepackage{enumitem}

% Graphics and figures
\usepackage{graphicx}
\usepackage[export]{adjustbox}
\graphicspath{{./images/}}
\usepackage{caption}
\usepackage{float}

% Tables
\usepackage{tabularx,makecell}
\usepackage{booktabs}
\usepackage[table,dvipsnames]{xcolor}
\usepackage{colortbl}
\usepackage{fancyhdr}
\usepackage{lastpage}
\usepackage{titlesec}
\renewcommand{\arraystretch}{1.5}
\definecolor{BrickRed}{RGB}{178,34,34}

% Code listings
\usepackage{listings}
\lstset{
    basicstyle=\ttfamily\small,
    backgroundcolor=\color{black!5},
    frame=single,
    framesep=5pt,
    breaklines=true,
    commentstyle=\color{gray}\itshape,
    keywordstyle=\color{blue}\bfseries,
    stringstyle=\color{red},
    showstringspaces=false,
    numbers=none,
    xleftmargin=10pt,
    xrightmargin=10pt
}

% Hyperlinks
\usepackage{hyperref}
\hypersetup{
    colorlinks=true,
    linkcolor=blue,
    filecolor=magenta,
    urlcolor=cyan
}
\urlstyle{same}

% Define a custom command for task labels that sets @currentlabel
\makeatletter
\newcommand{\tasklabel}[1]{%
  \phantomsection
  \def\@currentlabel{#1}%
  \label{#1}%
  #1%
}
\makeatother

% Header and Footer Configuration
\pagestyle{fancy}
\fancyhf{}
\fancyhead[L]{SentinAI-Ops}
% \fancyhead[C]{Center Header}
% \fancyhead[R]{Right Header}

\fancyfoot[L]{{\bf \color{black!30}AI for National Prosperity}}
% \fancyfoot[C]{\thepage}
\fancyfoot[R]{{\bf Page \thepage~of~\pageref{LastPage}}}

% Title Format Configuration
\titleformat{\section}
  {\normalfont\Large\bfseries\color{BrickRed}}
  {\thesection}{1em}{}

\titleformat{\subsection}
  {\normalfont\large\bfseries\color{ForestGreen!70}}
  {\thesubsection}{1em}{}

\setlength{\parindent}{0pt}
\setlength{\parskip}{10pt}
% Document metadata
\title{SentinAI-Ops: Technical Roadmap}
\author{Team CyberSentinel\\
AI for National Prosperity Hackathon\\
February 1 - March 15, 2026}
\date{}


\begin{document}
\begin{titlepage}

% Main title block (centered)
\begin{center}
    \vspace*{3cm}
    {\Huge \bfseries SentinAI-Ops \par}
    \vspace{2cm}
    {\Large \bfseries AI-Powered Anomaly Detection for Critical Infrastructure Defense \par}
    \vspace{0.5cm}
    % Accent line under title

    \end{center}

\vspace{2cm}

% Left-aligned team and hackathon info
\noindent
\renewcommand{\arraystretch}{2}
\begin{tabular}{ll}
\textbf{Submitted by:} & \\[-6mm]
&
\makecell[l]{Ben Tito - University of Nairobi\\ 
Emmanuel Nyabicha - Kenyatta University\\ 
Felix Robinson - Kenyatta University\\ 
Godfrey Ayuka - Kenyatta University\\
Shammah Sitati - JKUAT}\\

\textbf{Team Name:} & CyberSentinel\\

\textbf{Hackathon:} & AI for National Prosperity\\

\textbf{Category:} & Research Track – AI for Cybersecurity\\

% \textbf{Date:} & \\
\end{tabular}


\vfill
\begin{center}
\today
\end{center}


% \noindent\rule{0.5\linewidth}{0.5pt}\\
% \textit{This proposal is submitted for participation in the AI for National Prosperity Hackathon 2025.}

\end{titlepage}
%\maketitle
%\captionsetup{singlelinecheck=false}

\section{Project Overview}

\begin{center}
\begin{tabularx}{0.8\textwidth}{|l|X|}
\hline
%\rowcolor{BrickRed}
%\textcolor{white}{\textbf{Attribute}} & \textcolor{white}{\textbf{Details}} \\
%\hline
\rowcolor{black!20}
\bf Duration & 43 Days (6 Weeks) \\
\hline
\rowcolor{white}
\bf Team Size & 5 Members \\
\hline
\rowcolor{black!20}
\bf Total Tasks & 18 Tasks across 4 Phases \\
\hline
\rowcolor{white}
\bf Development Model & Parallel Track Architecture \\
\hline
\rowcolor{black!20}
\bf Roadmap Diagrams & \url{https://sentin-gantt.vercel.app/} \\
\hline
\end{tabularx}
\end{center}

\subsection*{Core Deliverables}
\begin{enumerate}
  \item Self-hosted Docker Compose deployment
  \item Three ML models (Isolation Forest, NLP Logs, Behavioral Profiling)
  \item Laravel Orchestrator with Blade + Livewire UI
  \item FastAPI AI Microservice for heavy computation
  \item Real-time threat visualisations (Livewire)
  \item Laravel Sanctum Authentication \& RBAC
\end{enumerate}

\section{Team Structure and Roles}

The team will be divided into two tracks (A and B) to enable parallel development and maximise efficiency. This separation allows both teams to work independently for the majority of the project timeline, converging only at critical integration points.

Track A focuses on backend infrastructure and machine learning models, while Track B concentrates on user interface and experience. This eliminates blocking dependencies, reduces development time by approximately $50\%$, and ensures that security-critical backend components remain isolated within the Docker internal network.

\subsection{Track A: Systems \& Intelligence (3 Members)}

\subsubsection*{Primary Responsibilities:}
\begin{enumerate}
  \item FastAPI AI Microservice development
  \item Machine Learning model implementation
  \item API Contract definition for Laravel consumption
  \item Prometheus and Loki configuration
  \item LLM context aggregation service
  \item Security hardening of internal ML endpoints
  \item Block Chain Integration
\end{enumerate}

\subsubsection*{Required Skills:}
\begin{enumerate}
  \item Python (FastAPI, scikit-learn, psutil, spaCy)
  \item Machine Learning and NLP
  \item DevOps (Docker, Prometheus, Loki)
  \item Blockchain (Hyperledger Fabric)
  \item System security and Linux administration
  \item Database Management (MySQL/PostgreSQL)
\end{enumerate}

\subsubsection*{Key Tasks:}
\begin{itemize}
  \item \ref{T1.2}: Backend API Foundation
  \item \ref{T1.4}: Monitoring Stack Configuration
  \item \ref{T2.1}: Model 1 - Isolation Forest
  \item \ref{T2.2}: Model 2 - NLP Log Analyzer
  \item \ref{T2.3}: Model 3 - Behavioral Profiling
  \item \ref{T3.1}: LLM Conversational Assistant
  \item \ref{T3.3}: Multi-Channel Alerting
  \item \ref{T3.5}: Security Hardening
\end{itemize}

\clearpage
\subsection{Track B: Orchestration \& Experience (2 Members)}

\subsubsection*{Primary Responsibilities:}
\begin{itemize}
  \item Laravel Application development (Orchestrator)
  \item Blade + Livewire real-time dashboard components
  \item Integration of FastAPI microservice endpoints
  \item User Management \& RBAC (Sanctum)
  \item Alerting System (Laravel Queues)
  \item Demo video production
\end{itemize}

\subsubsection*{Required Skills:}
\begin{itemize}
  \item PHP (Laravel 10/11), Livewire, Alpine.js
  \item UI/UX (Blade, Tailwind CSS)
  \item System Integration (Guzzle/HTTP Client)
\end{itemize}

\subsubsection*{Key Tasks:}
\begin{itemize}
  \item \ref{T1.3}: Frontend Dashboard Foundation
  \item \ref{T2.4}: Dashboard UI Components
  \item \ref{T3.2}: Chat Interface \& Remediation UI
  \item \ref{T4.3}: Demo Environment \& Video
\end{itemize}

\subsection*{Shared Responsibilities (Both Teams)}
\begin{itemize}
  \item \ref{T1.1}: Project Initialization (Day 1)
  \item \ref{T2.5}: API-Frontend Integration (Days 6-7)
  \item \ref{T3.4}: Authentication \& Authorization (Days 10-11)
  \item \ref{T4.1}: End-to-End Testing (Days 12-14)
  \item \ref{T4.4}: Documentation \& Guide (Days 13-14)
\end{itemize}

\section{Technology Stack}
\begin{center}
\begin{tabularx}{\textwidth}{|l|l|X|}
\hline
\rowcolor{BrickRed}
\textcolor{white}{\textbf{Layer}} & \textcolor{white}{\textbf{Technology}} & \textcolor{white}{\textbf{Purpose}} \\
\hline
\rowcolor{black!20}
Frontend/UI & Laravel Blade + Tailwind & Server-side rendered, responsive dashboard \\
\hline
\rowcolor{white}
Reactivity & Laravel Livewire & Real-time chart updates and live threat feeds \\
\hline
\rowcolor{black!20}
Orchestrator & Laravel (PHP 8.x) & Core logic: Auth, RBAC, Eloquent DB, Alert Queues \\
\hline
\rowcolor{white}
ML Engine & FastAPI (Python) & High-performance API for AI models (Isolation Forest, NLP) \\
\hline
\rowcolor{black!20}
Auth System & Laravel Sanctum & Secure session and token-based authentication \\
\hline
\rowcolor{white}
Alerting & Laravel Queues & Asynchronous multi-channel notifications (Telegram/SMS) \\
\hline
\rowcolor{black!20}
Metrics & Prometheus & Time-series database with PromQL \\
\hline
\rowcolor{white}
Logs & Loki + Promtail & Log aggregation for ML analysis \\
\hline
\rowcolor{black!20}
Audit \& Integrity & Hyperledger Fabric (Blockchain) & Immutable, tamper-evident ledger for server state changes, security events, and administrative actions \\
\hline
\end{tabularx}
\end{center}

\clearpage
\section{Development Timeline}

\subsection*{Gantt Chart Visualization}
\begin{center}
\includegraphics[width=0.9\textwidth]{0b3e96ce-3b36-41d4-8574-df272b66f20b-04_716_1707_1104_244}
\end{center}

\noindent
\textbf{Roadmap Diagrams:} \url{https://sentin-gantt.vercel.app/}

\subsection*{Phase 1: Foundation \& Infrastructure (Weeks 1-2: Feb 1-14)}

\noindent
\textbf{Objective:} Establish project infrastructure and enable parallel development

\begin{center}
\begin{tabularx}{\textwidth}{|l|p{4cm}|l|l|l|l|X|}
\hline
\rowcolor{BrickRed}
\textcolor{white}{\textbf{Task}} & \textcolor{white}{\textbf{Name}} & \textcolor{white}{\textbf{Track}} & \textcolor{white}{\textbf{Duration}} & \textcolor{white}{\textbf{Team}} & \textcolor{white}{\textbf{Dependencies}} & \textcolor{white}{\textbf{Deliverables}} \\
\hline
\rowcolor{black!20}
\rowcolor{black!20}
\tasklabel{T1.1} & Project Init & DevOps & 5 days & Full Team & None & GitHub repo, docker-compose skeleton, README \\
\hline
\rowcolor{white}
\tasklabel{T1.2} & FastAPI AI Microservice & Backend & 7 days & Track A & \ref{T1.1} & FastAPI structure, ML endpoints, Swagger docs \\
\hline
\rowcolor{black!20}
\tasklabel{T1.3} & Laravel Setup (Orchestrator) & Frontend & 7 days & Track B & \ref{T1.1} & Laravel 11, Livewire, Tailwind init, Database \\
\hline
\rowcolor{white}
\tasklabel{T1.4} & Monitoring Stack Config & DevOps & 10 days & Track A & \ref{T1.1} & Prometheus, Loki, exporters configured \\
\hline
\end{tabularx}
\end{center}

\subsubsection*{Milestones:}
\begin{itemize}
  \item Both teams working independently with clear interfaces
  \item docker-compose up runs all containers successfully
  \item Frontend uses mock data, backend connects to Prometheus/Loki
\end{itemize}

\subsection*{Phase 2: ML Models \& Core Features (Weeks 3-4: Feb 15-28)}

\noindent
\textbf{Objective:} Implement all ML models in parallel with dashboard components

\begin{center}
\begin{tabularx}{\textwidth}{|l|X|l|l|l|l|X|}
\hline
\rowcolor{BrickRed}
\textcolor{white}{\textbf{Task}} & \textcolor{white}{\textbf{Name}} & \textcolor{white}{\textbf{Track}} & \textcolor{white}{\textbf{Duration}} & \textcolor{white}{\textbf{Team}} & \textcolor{white}{\textbf{Dependencies}} & \textcolor{white}{\textbf{Deliverables}} \\
\hline
\rowcolor{black!20}
\tasklabel{T2.1} & Model 1: Isolation Forest & Backend & 10 days & Track A & \ref{T1.2}, \ref{T1.4} & Trained model, anomaly detection API \\
\hline
\rowcolor{white}
\tasklabel{T2.2} & Model 2: NLP Log Analyzer & Backend & 10 days & Track A & \ref{T1.2}, \ref{T1.4} & Log parsing engine, regex patterns \\
\hline
\rowcolor{black!20}
\tasklabel{T2.3} & Model 3: Behavioral Profiling & Backend & 14 days & Track A & \ref{T1.2} & psutil agent, process whitelist \\
\hline
\rowcolor{white}
\tasklabel{T2.4} & Blade/Livewire Components & Frontend & 14 days & Track B & \ref{T1.3} & System vitals, alert timeline, metrics viz \\
\hline
\rowcolor{black!20}
\tasklabel{T2.5} & Orchestrator-Engine Handshake & Both & 6 days & Both & \ref{T2.1}, \ref{T2.4} & Laravel consuming FastAPI endpoints \\
\hline
\end{tabularx}
\end{center}

\subsubsection*{Milestones:}
\begin{itemize}
  \item All three ML models operational with test data
  \item Dashboard displays real backend data
  \item End-to-end anomaly detection functional
\end{itemize}

\clearpage
\subsection*{Phase 3: Intelligence \& User Experience (Week 5: Mar 1-7)}

\noindent
\textbf{Objective:} Add AI assistant, authentication, alerting, and security

\begin{center}
\begin{tabularx}{\textwidth}{|l|X|l|l|l|l|X|}
\hline
\rowcolor{BrickRed}
\textcolor{white}{\textbf{Task}} & \textcolor{white}{\textbf{Name}} & \textcolor{white}{\textbf{Track}} & \textcolor{white}{\textbf{Duration}} & \textcolor{white}{\textbf{Team}} & \textcolor{white}{\textbf{Dependencies}} & \textcolor{white}{\textbf{Deliverables}} \\
\hline
\rowcolor{black!20}
\tasklabel{T3.1} & LLM Conversational Assistant & Backend & 7 days & Track A & \ref{T2.1}, \ref{T2.2}, \ref{T2.3} & Flask aggregator, LLM integration \\
\hline
\rowcolor{white}
\tasklabel{T3.2} & Chat Interface & Frontend & 7 days & Track B & \ref{T2.4}, \ref{T3.1} & AI chat component, remediation display \\
\hline
\rowcolor{black!20}
\tasklabel{T3.3} & Multi-Channel Alerting (Jobs) & Backend & 7 days & Track B & \ref{T2.5} & Laravel Queues for Telegram, Email \\
\hline
\rowcolor{white}
\tasklabel{T3.4} & Auth \& RBAC (Sanctum) & Security & 7 days & Track B & \ref{T1.3} & User roles, protected routes \\
\hline
\rowcolor{black!20}
\tasklabel{T3.5} & Security Hardening & Security & 7 days & Track A & \ref{T1.1} & Network isolation, secret management \\
\hline
\end{tabularx}
\end{center}

\subsubsection*{Milestones:}
\begin{itemize}
  \item AI assistant provides contextual threat explanations
  \item Multi-channel alerts operational
  \item Secure admin access implemented
  \item Backend isolated within Docker network
\end{itemize}


\clearpage
\subsection*{Phase 4: Integration, Testing \& Demo (Week 6: Mar 8-15)}

\noindent
\textbf{Objective:} End-to-end testing, attack simulation, demo preparation

\begin{center}
\begin{tabularx}{\textwidth}{|l|X|l|l|l|l|X|}
\hline
\rowcolor{BrickRed}
\textcolor{white}{\textbf{Task}} & \textcolor{white}{\textbf{Name}} & \textcolor{white}{\textbf{Track}} & \textcolor{white}{\textbf{Duration}} & \textcolor{white}{\textbf{Team}} & \textcolor{white}{\textbf{Dependencies}} & \textcolor{white}{\textbf{Deliverables}} \\
\hline
\rowcolor{black!20}
\tasklabel{T4.1} & End-to-End Testing & QA & 6 days & Full Team & \ref{T3.1}-\ref{T3.5} & Test scenarios, bug fixes \\
\hline
\rowcolor{white}
\tasklabel{T4.2} & Attack Simulation & Security & 4 days & Track A & \ref{T2.3} & Reverse shell test, detection validation \\
\hline
\rowcolor{black!20}
\tasklabel{T4.3} & Demo Environment & Demo & 6 days & Track B & \ref{T4.1} & Demo script, video recording \\
\hline
\rowcolor{white}
\tasklabel{T4.4} & Documentation & Docs & 4 days & Full Team & \ref{T4.1} & Installation guide, architecture diagrams \\
\hline
\end{tabularx}
\end{center}

\subsubsection*{Milestones:}
\begin{itemize}
  \item Detection-to-alert latency under 5 seconds
  \item Professional demo video completed
  \item Complete installation documentation
  \item Zero critical bugs
\end{itemize}

\section{Development Timeline \& Weekly Routine}

The project follows a 6-Week Agile Sprint structure. Instead of a rigid daily timetable, the team operates on a cyclical weekly routine that balances deep work with necessary synchronization points.

\subsection*{Sunday: Sprint Planning \& Strategy}
\begin{itemize}
  \item Full Team Virtual Meeting (1 Hour): The week begins with a high-level sync to review the backlog and define the specific goals for the upcoming cycle.
  \item Role Assignment: Tasks are distributed based on the current priorities. Track A (Backend) defines the necessary API contracts and data models, while Track B (Frontend) confirms that the proposed data structures meet the UI visualization requirements.
\end{itemize}

\subsection*{Monday \& Tuesday: Parallel Execution}
\begin{itemize}
  \item Mode of Work: Asynchronous development.
  \item Track A Focus (Backend): Deep technical implementation, including infrastructure setup, database schema management, and ML model training.
  \item Track B Focus (Frontend): UI/UX development, creating layout scaffolding, and building visual components using mock data until the backend is ready.
\end{itemize}

\subsection*{Wednesday: Mid-Week Checkpoints}
\begin{itemize}
  \item Track-Specific Virtual Meetings (\textbf{30} Minutes): Each track holds a short, focused call to address specific technical blockers.
  \item Technical Review: Track A reviews code for endpoint logic and model accuracy. Track B reviews the interface responsiveness and theme consistency.
\end{itemize}

\subsection*{Thursday \& Friday: Deep Work \& Integration}
\begin{itemize}
  \item Heavy Coding: Teams focus on complex logic implementation and connecting the separate components.
  \item Integration (Friday): Code branches are merged into the development environment. Track A focuses on testing Docker container stability, while Track B performs end-to-end flow testing to ensure the dashboard correctly displays real-time data from the API.
\end{itemize}

\subsection*{Saturday: Review \& Deliverables}
\begin{itemize}
  \item Full Team Physical Meeting (3 Hours): The week concludes with a comprehensive review of the output.
  \item System Demo: The team walks through the completed features. Track A verifies system security and latency, while Track B verifies the user journey and aesthetics.
  \item Next Steps: Any incomplete items are moved to the backlog for the following Sunday's planning session.
\end{itemize}

\newpage
\section{Architecture Diagrams}

\subsection{System Architecture Flowchart}
\begin{center}
\includegraphics[width=0.9\textwidth]{0b3e96ce-3b36-41d4-8574-df272b66f20b-09_1711_1754_552_249}
\end{center}

Roadmap Diagrams: \url{https://sentin-gantt.vercel.app/}

\subsection{Parallel Development Workflow}
\begin{center}
\includegraphics[width=0.9\textwidth]{0b3e96ce-3b36-41d4-8574-df272b66f20b-10_2156_1687_328_249}
\end{center}

\clearpage
\section{Quick Reference}

\subsection{Installation Commands}

\begin{lstlisting}[language=bash]
# Clone repository
git clone https://github.com/Nuelnyakyz/SentinAI-Ops.git
cd SentinAI-Ops

# Configure environment
cp .env.example .env

# Launch stack
docker-compose up -d

# Access dashboard
open http://localhost:8501
\end{lstlisting}

\subsection{Key Endpoints}
\begin{itemize}
  \item Dashboard: \href{http://localhost}{http://localhost}
  \item FastAPI Microservice: \href{http://localhost:8000/docs}{http://localhost:8000/docs}
  \item Prometheus: \href{http://localhost:9090}{http://localhost:9090}
  \item Loki: \href{http://localhost:3100}{http://localhost:3100}
\end{itemize}

\vspace{1em}
\noindent
\textit{Hackathon: AI for National Prosperity 2026}


\end{document}